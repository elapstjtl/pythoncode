\documentclass[UTF8]{ctexart}
\usepackage{CJK}
\usepackage{listings}
\usepackage{color,xcolor}
\newtheorem{thm}{定理}%定义一个定理环境,有一个可选参数,即定理的名字
%在定义时这个参数用{}表示定理名的前缀,但在使用时参数用[]
%  同时会计数
\newenvironment{myquote}{\begin{quote}\zihao{-5}\kaishu}{\end{quote}}
%定义myquote环境,第一个参数是环境开始时代码,第二个参数是结束时代码,
%将这两部分代码嵌入至新环境中,当要更改时直接在此处更改即可
\newcommand\degree{^\circ}
%定义新命令,newcommand的两个参数分别是新命令和新命令的定义

\usepackage{graphicx}%插图的宏包
\usepackage{float}%浮动体环境
\usepackage{amsmath}%有公式引用命令\eqref
\usepackage{geometry}%可以设置页面尺寸
\usepackage{txfonts}
\geometry{a4paper,left=2.5cm,right=2.5cm,bottom=2.5cm,top=2.5cm}%A6纸大小,版心居中,长宽占页面0.8倍
\usepackage[format=hang,font=small,textfont=it]{caption}
%所有标题使用悬挂对齐方式(编号相左对齐,整体用小字号,而标题文本用斜体(对汉字来说就是楷书)
\usepackage[nottoc]{tocbibind}%宏包默认在目录中加入目录项本身、参考文献、索引等项目
%用Nott选项取消加入目录项本身
\usepackage[justification=centering]{caption} %caption可设置标题格式
\usepackage[center]{titlesec}%标题

    \usepackage{amssymb}
    \usepackage{abstract}
    \usepackage{graphicx}
    \usepackage{caption}
    \usepackage{subfigure}
    \usepackage{fancyhdr}
    \usepackage{booktabs}%toprule和midrule命令的宏包
  
    \usepackage{geometry}
\usepackage{mathptmx}%公式字体
    \definecolor{mygreen}{rgb}{0,0.6,0}
    \definecolor{mygray}{rgb}{0.5,0.5,0.5}
    \definecolor{mymauve}{rgb}{0.58,0,0.82}
    \lstset{
backgroundcolor=\color{white},   % choose the background color
 % size of fonts used for the code或改成\small\monaco稍大
numbers=left,                        % 设置行号
           % 设置行号字体大小
columns=fullflexible,
breaklines=true,                 % automatic line breaking only at whitespace
captionpos=b,                    % sets the caption-position to bottom
tabsize=4,                       % 把tab扩展为4个空格,默认是8个太长
commentstyle=\color{mygreen},    % 设置注释颜色
escapeinside={\%*}{*)},          % if you want to add LaTeX within your code
keywordstyle=\color{blue},       % 设置keyword颜色
    % string literal style
frame=single,                        % 设置有边框
rulesepcolor=\color{red!20!green!20!blue!20},
% identifierstyle=\color{red},
language=matlab,
}
\usepackage[font={sf,bf},textfont=md]{caption}
%\usepackage{caption}
\pagestyle{empty}
\newcommand{\normal}{\fontsize{10pt}{1.2pt}\selectfont}
\newcommand{\sanhao}{\heiti\zihao{3}}
\newcommand{\sihao}{\heiti\zihao{4}}
\newcommand{\xiaosihao}{\heiti\zihao{-4}}
\newcommand{\wuhao}{\songti\zihao{-5}}
\newcommand{\songsihao}{\songti\zihao{-4}}
\newcommand{\Engsihao}{\heiti\zihao{-4}}
\titleformat*{\section}{\sihao\centering}
\titleformat*{\subsection}{\xiaosihao}
\titleformat*{\subsubsection}{\xiaosihao}
\titleformat*{\paragraph}{\songti\zihao{-4}}
\titleformat*{\subparagraph}{\songti\zihao{-4}}
%\setmainfont{Times New Roman}
\linespread{1.25}
\bibliographystyle{plain}
\pagestyle{fancy}
% 设置英文字体
%\setmainfont{Times New Roman}
%\newcommand{\tm}{\fontspec{Times New Roman}}



\usepackage{caption}
%设置图片标题字体格式
\captionsetup{font=small}

\begin{document}
\songsihao
\begin{center}
    \sanhao{基于回归分析的污染源位置确定模型}
   
\end{center}

\section{摘要}
针对问题一,题目指出污染物浓度随距离的衰减函数为指数函数关系,根据题目所给数据, 
对污染物的扩散函数进行{\bf 线性化},再通过\emph{\bf 线性最小二乘法}进行\emph{\bf 线性回归},求出了实验数据
扩散函数为  $C=100.0092 e^{-0.020011r}$ ,对自然下的污染物扩散,
引入自然因素作用方向$\theta$及作用强度$\rho$,得到了扩散函数。\\
针对问题二,首先给出了普通\emph{\bf 非线性最小二乘}的思想,其次由于题目数据的特殊性,
给出了求解问题的两种方案,方案一假设每个方向的扩散无差异性,其次提出了\emph{\bf 加权
最小二乘法},借助\emph{\bf 高斯牛顿迭代法}求出了加权最小二乘法的解为污染源位置\textit{
 (-2.8978,4.4011)} 污染物浓度为  \textit{182.7441} ,
方案二考虑了扩散的差异性,首先通过摒弃自然因素作用的两个参数以及一个可能异常
的观测点通过\emph{\bf 牛顿迭代法}求出污染源信息,其次重新考虑摒弃的参数及观测点,
解出了污染源及自然因素的作用强度信息,污染源位置\textit{(-1.4180,4.1000)}, 污染物浓度为\textit{199.5436}  ,
最后对自然因素的作用方向进行了\emph{\bf 灵敏度分析},发现结果具有较强的稳定性。
\\
针对问题三,给出了求解多点污染源信息的一般解法,考虑了不同污染源的污染物
扩散过程引起的相互影响问题,引入了不同污染源之间的\emph{\bf 交叉项},并分析
了不同污染源影响情况与扩散公式的对应关系。最后,给出了对不同情况下如何对扩散
函数进行求解的方案。囿于题目数据,没有再进行求解。\\
{\bf 关键词:}线性化 \quad 最小二乘法\quad 加权最小二乘法\quad 高斯牛顿迭代法\quad 牛顿迭代法
\newpage
    \section{问题重述}
      \subsection{问题一}
         对实验室测定的一组数据,在已知污染物浓度随距离
         的关系为指数函数的情况下,模拟得到该污染物的扩散情况。
      \subsection{问题二}
         已知平面上四个点的位置及对应的污染物浓度,根据问题一得到的扩散
         情况,大致确定污染源的位置及污染物浓度。
      \subsection{问题三}
         更一般的,当存在多点污染源时,如何解出不同污染源的位置及
         污染物浓度,此时,又需要测定多少数据。
         
\section{模型假设}
  \begin{enumerate}
    \item 所有测定都保证污染物浓度已经相当稳定,即不考虑时间因素
    \item 所有的测定都没有非常大的误差(允许误差但无异常值)
    \item 不考虑传播过程的介质改变问题(当然也不考虑障碍物)
    
  \end{enumerate}
\section{符号说明}
  \begin{table}[htbp]
    \centering
    \caption{符号说明表}
    \begin{tabular}{c c}
      \toprule[1.5pt]
      符号 & 意义\\
      \midrule[1pt]
      $ C_0 $ & 污染源污染物浓度\\
      $ (x_0,y_0) $ & 污染源坐标\\
      $ C $ & 任意点的污染物浓度\\
      $ r_i $ & 实验测得的第 i 组数据的距离\\
      $ C_i $ & 实验测得的第 i 组数据的浓度\\
      $ (x_i,y_i) $ & 平面内第 i 个点的位置\\
      $ D_i $ & 实际测得的第 i 个点的浓度\\
   
      \bottomrule[1.5pt]
    \end{tabular}
  \end{table}
       
\newpage
    
\section{问题分析}
    \subsection{问题一}
      针对问题一,由于题目指出污染物浓度随距离的衰减函数为指数函数关系,
      因此考虑对题目所给数据进行参数估计(指数型拟合),
      能够得出当污染源的浓度为题目所给数据(100.0)时,
    污染物浓度随时间的关系,由于题目没有给出更多的数据,
    当污染源浓度为不同值时,无法确定扩散函数是否具有相同的趋势,
    为方便分析,这里假设对不同的污染源浓度,其扩散函数线性相关。
    进而可以得到理想状况下不同污染源浓度的扩散函数。
    而对于现实中的污染物扩散问题,由于自然因素,在不同方向上的
    扩散可能有所不同,这里假设由自然因素造成的影响是单方向的,
    因此考虑在原扩散函数的不同方向前加扩散系数即可。
    \subsection{问题二}
      在得到自然条件下的扩散函数后,可以对测得的数据进行拟合。
      由于题目所给数据不够充足,观测数目小于参数个数,因此给出两种
      方案:\\
      方案一:不考虑不同方向的扩散差异,进行非线性回归\\
      注意到题目所给数据中,A点和D点具有相同的污染物浓度,
      如果所给数据完全精确,污染源应该位于线段AD的中垂线上,
      那么B点和C点应该有相同的污染物浓度,但实际上B点与C点
      污染物浓度相差较大,因此可以认定污染物浓度的测定有误差。
      原问题即转化为寻一点和一初始浓度,使所有数据点的实际污染物浓度
      与理论的污染物浓度的误差最小。通过多元非线性回归可以得到误差最小
      时的初始浓度及污染源位置。\\
      方案二:考虑不同方向的扩散差异,大概估计一个参数,对余下数据求解\\
      根据题目数据大概估计一个参数,余下的数据是一个有唯一解的方程组,可以
      求出污染源位置。最后,可以对估计的参数进行灵敏度分析。
    \subsection{问题三}
      当分析更多的点污染源时,两污染源传播中会发生相遇,
      当不考虑相遇造成的影响时,可以认为各点的污染物浓度是简单的
      叠加,如果再考虑相遇造成的影响,可以加上污染源引起的交叉项。
      重新进行多元非线性回归,可以得到各个点的初始浓度及污染物浓度
\newpage
\section{模型建立}
 \subsection{问题一}
   设污染物浓度随时间变化关系为
   $$ C=e^{ar+b} $$
   两边取对数得
   $$\ln{C}=ar+b$$
   通过最小二乘法进行线性回归,求
   $$f(a,b)=\sum_{i = 0}^{10}  {(ar_i+b-\ln{C_i})^2}$$
  
   $$  \min{f(a,b)}$$
   
   根据二元函数极值的求法,求
   \begin{equation*}
  \left\{
    \begin{array}{rc}\frac{\partial f}{\partial a}=0\\
   \frac{\partial f}{\partial b}=0 \end{array}\right.
    \end{equation*}
  得正规方程组
   \begin{equation}\label{p1.1}
    \left\{
      \begin{array}{rcl}
        a \sum\limits_{i =0}^{10}{r_i^2}+b \sum\limits_{i =0}^{10}{t_i}=
        \sum\limits_{i=1}^{10}{r_i\ln{C_i}}\\
        a \sum\limits_{i =0}^{10}{r_i}+11b=\sum\limits_{i=0}^{10}{\ln{C_i}}
      \end{array}
    \right.
   \end{equation}
   解得 $a$、$b$ 的值后,在理想情况下,
   对污染源污染物浓度为 $C_0$ ,则平面内某点 $(x,y)$ 的污染物浓度
   \begin{equation}\label{p1.2}
    C=C_0e^{a\cdot \sqrt{(x-x_0)^2+(y-y_0)^2}} 
   \end{equation}
   在单方向的自然因素下,设作用方向与 x 轴夹角为 $\theta$ ($0\leq \theta<\pi$)
   ,作用强度为 $\rho $, 则
   \begin{equation}\label{p1.3}
    C=C_0 e^{a \sqrt{(\rho\cos{\theta}+1)(x-x0)^2+(\rho\sin{\theta}+1)(y-y_0)^2}}
   \end{equation}
  \subsection{问题二}
  

    \subsubsection{方案一的建立}
    首先对于多参数非线性模型,根据普通最小二乘原理给出参数估计的一般方法。
    \\
    设 $\emph{X}$ 是给出的自变量矩阵,$\emph{Y}$ 是 $\emph{X}$ 对应的因变量列向量,
    $\emph{f}$ 是需要拟合的函数,$ \beta$ 是需要确定的参数列向量,${\varepsilon}$
    是对应的误差, 则有
    $$ Y=f(X,\beta)+\varepsilon $$
    应使残差平方和最小,即
    \begin{equation}\label{aim}
       \min{ \{\varepsilon ^{T} \cdot \varepsilon\}} 
    \end{equation}
       求
    \begin{equation}
      \frac{\partial}{\partial \beta } (f(X,\beta)^{T})
      \cdot (Y-f(X,\beta))={0}
    \end{equation}
    (0是与$\beta $的长度相同的0向量)\\
    事实上,可以通过高斯-牛顿迭代法求得\ref{aim}的极小值点。
    

    由于每一个观测值的不确定性有所不同,需要对不同观测的
    偏差进行加权\cite{zhuangshijian},以反应不同观测数据之间的相对重要性。
    加权的基本思想可以描述为:如果已知观测值 $y_k$ 的不确定度高于
     $y_l$ ,那么权值 $\omega_k$ 就应该低于 $\omega_l$ 。\cite{key}
    \\
    若观测值 $y_i$ 的概率分布函数中的标准不确定度为 $\sigma$ ,则
    最优加权值应该为:
     $$\omega_i=\frac{1}{\sigma_i ^2}$$
    \\
    由于无法直接获得 $\sigma$ ,这里考虑用前面一次拟合的残差的平方项代替 $\sigma_i^2$来构造权值,
    即有
    $$ \Delta_i =y_i-\hat{y} $$
    $$\omega_i=\frac{1}{{\sigma_i}^2}\approx \frac{1}{\Delta _i^2} $$
    另外,为避免小误差对权重的影响(此时权重将很大,不同观测点的相对重要性无法很好体现),
    可以设置下界$\lambda_L$来实现\cite{key},即
    \begin{eqnarray} %\label{p2.3},不能加标签,但可在内部加标签
      \omega_i=&=&%还不知道有啥用
      \left\{
      \begin{array}{lll}\label{p2.3}
         \frac{1}{\lambda_L^2}
    \mbox{,当} |\Delta_i|<\lambda_L \mbox{时}\\
     \frac{1}{\Delta_i^2} \ \mbox{,其他}
      \end{array}\right. 
    \end{eqnarray}
    \ref{p2.3}中的门限值 $\lambda_L$ 可以由下式定义:\cite{key}
    \begin{equation}\label{p2.4}
      \lambda_L=\kappa\hat{\sigma}_y
    \end{equation}
    $\kappa$ 为与应用背景相关因子,因为$\kappa$取很小门限值 \ref{p2.4} 
    将失去意义,可以考虑在一定范围内对$\kappa$取不同的值分别求出拟合情况
    $\hat{\sigma}$ 为观测值 $y_i$ 标准
    差的估计值,其表达式为
    \begin{equation}\label{guji}
       \hat{\sigma}_y=\sqrt{\frac{1}{N-M}\cdot\frac{\sum
    \omega_i(\hat{y_i}-y_i)^2}{\frac{1}{N}\cdot \sum{\omega_i}}}
    \end{equation}
    其中 N 为数据个数,M 为模型参数个数。可以通过上一次拟合的权值以及偏差
    根据$\ref{p2.3}$以及$\ref{guji}$求出下一次的权值。
    得到权值 $\omega$ 后,原普通最小二乘$\ref{aim}$ 就变为

    \begin{equation}\label{aim2.2}
      \min{\{\varepsilon^T\cdot diag(\omega) \cdot \varepsilon\}}
    \end{equation}
    上式即为
    \begin{equation}
      \min{\sum_{i=1}^n {(\sqrt{\omega_i}\cdot\varepsilon_i)^2}}
    \end{equation}
    同样可以用高斯牛顿迭代求解该式。\\
    另外,可以通过相关指数法判断加权后的回归方程效果,其表达式为:
    $$ R^2=1-\frac{\sum\limits_{i=1}^n \omega_i(y_i-\hat{y_i})^2}{\sum\limits_{i=1}^n
     \omega_i(y_i-\overline{y_i})^2}$$
    $R^2$ 越接近1说明拟合效果越好\\
    综合以上,如果通过普通最小二乘法得到的残差平方和很大,可以通过加权最小
    二乘法寻找权重,
    寻求最优权重的方法可由下面得到:
    \begin{enumerate}\label{solve2.1}
      \item 初始化权值为等权值,并计算出此时的参数及偏差
      \item 初始化$\kappa$,根据 \ref{p2.4}及 \ref{guji}求出新的权重
      \item 由新权重根据 $\ref{aim2.2}$ 求出新参数及权重下的相关指数和残差平方和
      \item 重新设定 $\kappa$,回到2.。
      \item 比较不同$\kappa$ 下的相关指数、残差平方和。
    \end{enumerate}

  根据\ref{p1.2}对平面内某点 $(x,y)$ ,污染物浓度与位置的经验公式
     \begin{equation}\label{p2.1}
        \hat{C}=\hat{C_0} e^{a\sqrt{(x-\hat{x_0})^2+(y-\hat{y_0})^2}}
     \end{equation}
  考虑用最小二乘法进行非线性回归
    $$ \hat{C_0} e^{a \sqrt{(x_i-\hat{x_0})^2+(y_i-
    \hat{y_0})^2}}=D_i+\varepsilon_i $$
  跟据之前的结论可以求解。
  \subsubsection{方案二的建立}
  由于题目只有四组观测值,\ref{p1.3}有五个参数,考虑首先对某个参数进行估计,
  注意到第三组数据有显著不同,而其余三组相差很小,可以考虑首先对自然因素的作用方向 $\theta$ 进行估计。
  具体做法如下:
 \begin{itemize}\label{p2.2.1}
   \item 摒除第三组数据,并不考虑扩散系数,根据\ref{p1.2}对余下方程求解,得到污染源位置及浓度数据
   \item 考虑扩散系数,将第三组数据与任意一组数据(这里用第二组)联立,并将之前得到的污染源信息代入,
   根据\ref{p1.3}求解,得到作用强度及作用方向
   \item 考虑扩散系数,将四组数据联立,将得到的作用方向代入方程,重新根据\ref{p1.3}求解,得到新的污染源信息
   及作用强度
 \end{itemize}
 以上求解方程可通过牛顿迭代法实现。
  \subsection{问题三}
  首先给出确定多点污染源位置的一般解法,对 n 个点污染源,
  设其位置分别为
  $$(x_i,y_i) , i=1,2,3\cdots ,n$$ 
  污染源浓度为
  $$ C_i , i=1,2,3\cdots,n $$
  则当不考虑不同污染源间的相互影响,对平面内任一点$(x,y)$有
  \begin{equation}\label{p3.1}
    C=\sum\limits_{i=1}^{n}{C_i e^{a\cdot \sqrt{k_{xi}(x-x_i)^2+k_{yi}(y-y_i)^2}}}
  \end{equation}
  其中 $k_{xi}$ $k_{yi}$ 分别为第 i 个污染源在 $x$ 和 $y$ 方向的扩散系数。\cite{gaozhen}\\
  现考虑不同污染源之间的叠加,加入不同污染源之间的交叉项
  \begin{equation}\label{p3.2}
    C=\sum\limits_{i=1}^{n}{C_i e^{a\cdot \sqrt{k_{xi}(x-x_i)^2+k_{yi}(y-y_i)^2}}}
    +\sum\limits_{i<j\leq n}{P_{ij}C_i e^{a\cdot \sqrt{k_{xi}(x-x_i)^2+k_{yi}(y-y_i)^2}}
    \cdot C_j\cdot e^{a\cdot \sqrt{k_{xj}(x-x_j)^2+k_{yj}(y-y_j)^2}}}
  \end{equation}
  当 $P_{ij}<0$ 时,表明第i与第j个污染源点有相互抑制作用,\\
  当 $P_{ij}>0$ 时,表明第i与第j个污染源点有相互促进作用,\\
  当 $P_{ij}=0$ 时,表明第i与第j个污染源点无作用,\\
  考虑模型带来的误差
    $$\varepsilon=D-C $$
  综合考虑上述多污染源点模型,既能体现单污染源点之间的叠加,
  又能体现不同污染源之间的促进与抑制作用。最后可以通过之前得到的结论求解。\\
  特别地,当只有两个污染源点时
  \begin{equation}\label{p3.3}
    C=\sum\limits_{i=1}^{2}{C_i e^{a \cdot \sqrt{k_{xi}(x-x_i)^2+k_{yi}(y-y_i)^2}}}
    +P_{12}C_1\cdot C_2 e^{a \cdot \sqrt{k_{x1}(x-x_1)^2+k_{y1}(y-y_1)^2}}\cdot 
    e^{a\cdot \sqrt{k_{x2}(x-x_2)^2+k_{y1}(y-y_2)^2}}
  \end{equation}
  该式有11个参数,当观测点有11个时,原问题是可解方程,用牛顿迭代法即可。\\
  当观测点数目大于11个时,可以通过最小二乘法\ref{aim}求得参数。\\
  如果最终求得的残差的平方和很大,说明不应该同等对待各个观测点,可以通过加权
  最小二乘法\ref{aim2.2}及\ref{solve2.1}求得。\\
  由于题目只给了四个观测点,小于拟合的参数个数,不再进行求解。
\newpage
\section{模型求解}
  \subsection{问题一} 
  通过matlab对\ref{p1.1}进行求解并绘出了拟合曲线\\
  \begin{figure}[htbp]
    \centering
    \includegraphics[scale=0.7]{p(1.1).jpg}
    \caption{对数据的拟合图像}
    \label{pp1.1}
  \end{figure}
 \\
  拟合曲线方程为
  $$C=100.0092 e^{-0.020011r}$$
从而当污染源浓度为 $C_0$时
其扩散函数为
\begin{equation}\label{p3.2.1}
 C=C_0 e^{-0.020011\cdot \sqrt{(x-x_0)^2+(y-y_0)^2}}
\end{equation}
 考虑自然因素,扩散函数为
\begin{equation}\label{p3.2.2}
  C=C_0 e^{-0.020011 \sqrt{(\rho\cos{\theta}+1)(x-x0)^2+(\rho\sin{\theta}+1)(y-y_0)^2}}
 \end{equation}

 
  \subsection{问题二}
  
  \subsubsection{对方案一的求解}
  首先跟据普通最小二乘法,设置权重为等权重,
     通过matlab用高斯牛顿迭代对\ref{aim}进行求解,解得
     $$ x_0=-2.4913,y_0=5.5437$$
     $$ C_0=180.9926 $$
     此时的残差平方和为 $578.1119$ ,相关指数 $R_1^2=0.5528$,
     说明用等权值进行回归效果不理想 ,进而继续根据$\ref{solve2.1}$
     取 $\kappa$ 在 $0.525-0.925$ 区间内分别求出新权重,
     并求出该权重下对应的拟合参数,相关指数 $R^2$ 及残差平方和 r。
     \begin{table}[htbp]
      \caption{最终数据}
     \begin{tabular}{|c|c|c|c|c|c|c|}
      \hline  $\kappa$ &0.425 & 0.525 &0.625 &0.725 &0.825 &0.925 \\
      \hline  $R^2$    &
      0.9523  &  0.8927  &  0.7834 &   0.5528   & 0.5528  &  0.5528   \\
      \hline  $r$    & 366.7063 & 461.2554 & 532.7248 & 578.1119 & 578.1119 & 578.1119   \\
      \hline  $(x_0,y_0)$ & (-2.897,4.401)   &(-2.341,4.615)  & (-2.150,5.022)  & (-2.491,5.543)  
       &(-2.491,5.543)  & (-2.491,5.543) \\
      \hline  $C_0$ &
      182.7441  & 182.4612 & 181.7364 & 180.9926 & 180.9926 & 180.9926 \\
    \hline
    \end{tabular}
    \end{table}
    \\
    当 $\kappa$ 大于等于0.725时,效果与等权值拟合一致。当$\kappa$ 最小
    等于0.425时具有最好的拟合结果,此时的四个观测点的权值分别为 1.1748, 0.7003,0.8656,1.1748,
    污染源位置为$(-2.897,4.401)$,污染物浓度为 $182.7441$。\\
    对不同组的污染源信息分析也可发现,尽管应当对不同的观测点
    赋予不同的权重,但加权最小二乘与普通最小二乘的结果并没有很大
    区别,这点与文献\cite{key}得出的结论一致。
    \\
     并绘出了此时浓度随位置关系的三维图以及等值线图
     \begin{figure}[htbp]
      
              \centering
              \includegraphics[scale=0.65]{p2.1.1.jpg}
              \caption{三维图}
              \label{pp2.1.1}
       \end{figure}
      
       \begin{figure}
            \centering
            \includegraphics[scale=0.65]{p2.1.2.jpg}
            \caption{等值线图}
            \label{pp2.1.2}
    
          
     \end{figure}
     \newpage
     \subsubsection{对方案二的求解}  
     根据\ref{p2.2.1}  解出了最终的污染源位置及自然因素的作用情况,结果如下
 $$
  (x_0,y_0)=(-1.4180,4.1000), \ C_0=199.5436 $$
$$ \rho=  2.8167,\ \theta=  1.5854 $$
绘出了其三维图及等值线图:
\begin{figure}[htbp]
  \centering
  
          \includegraphics[scale=0.65]{p2..2.1.jpg}
          \caption{三维图}
          \label{pp2.1}
 \end{figure}
\begin{figure}
  
        \centering
        \includegraphics[scale=0.65]{p2.2.2.jpg}
        \caption{等值线图}
        \label{pp2.2}
  
      
 \end{figure}
\newpage
\section{模型检验及灵敏度分析}
只对方案二的求解进行灵敏度分析:\\
改变通过\ref{p2.2.1} 解出的自然因素的作用方向$\theta$,考察此时污染源信息及作用强度随该
作用方向的变化情况。
以下是分别的关系图:
\begin{figure}[htbp] %图片标注
  \centering
  \subfigure*
  {
      \begin{minipage}[t]{0.4\linewidth}
          \centering
          \includegraphics[scale=0.36]{lingx_0.jpg}
          \caption{横坐标$x_0$的影响图}
          \label{1(1)}
      \end{minipage}
  }
  {
      \begin{minipage}[t]{0.4\linewidth}
          \centering
          \includegraphics[scale=0.36]{lingy_0.jpg}
          \caption{纵坐标$y_0$的影响图}
          \label{1(2)}
      \end{minipage}
  }
  {
      \begin{minipage}[t]{0.4\linewidth}
          \centering
          \includegraphics[scale=0.36]{lingC_0.jpg}
          \caption{污染源浓度的影响图}
          \label{2(1)f}
      \end{minipage}
  }
  {
      \begin{minipage}[t]{0.4\linewidth}
          \centering
          \includegraphics[scale=0.36]{lingrho.jpg}
          \caption{作用强度的影响图}
          \label{2(1)c}
      \end{minipage}
  }
    \end{figure}
    \newpage
    可以发现作用方向在 $\theta=1.5854$ 附近,
    污染源信息以及作用强度都没有明显变化,以下是在其周围的部分数据:\\
    \begin{figure}[htbp]
      \centering
      \includegraphics[scale=0.7]{partdata.png}
      \caption{部分数据}
      \label{fig:partdata}
    \end{figure}
    \\
    可以发现,$x_0$ 大约在-1至-2.5 之间,$y_0$ 大约在3至5之间
    $C_0$ 大约在190至220之间, $\rho$ 大约在3至5之间。
    都没有太大的变化。因此可以认定原解具有较好的稳定性。
\section{模型评价}
\subsection{优点}
\begin{enumerate}
  \item 给出了单点污染源确定的一般方法,并且基于题目数据进行了一些调整
  \item 考虑了不同观测点的重要性,利用了加权最小二乘法,使结果更准确
  \item 给出了求解多点污染源信息的一般解法
\end{enumerate}
\subsection{缺点}
\begin{enumerate}
  \item 方案一的权重估计并不严格,寻求更优权重可见迭代重加权最小二乘法\cite{yuxiaoyan}
\end{enumerate}

  \bibliography{in.bib}

\section{附录}
以下为方案一的求解:
\begin{lstlisting}
  clear
Data=xlsread('data.xlsx');%读取数据

d_dis=Data(1,:);
d_consis=Data(2,:);
scatter(d_dis,d_consis);
hold on;
y=log(d_consis);
fit=polyfit(d_dis,y,1);
fit(2)=exp(fit(2));
xfit=min(d_dis):0.5:max(d_dis);
yfit=fit(2)*exp(fit(1)*xfit);
h=plot(xfit,yfit);
legend("数据点","拟合曲线");
set(gca,'fontsize',15);
title("对数据的拟合图");
str=['C=' num2str(fit(2)) 'e^{' num2str(fit(1)) 'r}'];%设置字符串
h2=text(50,60,{str});%绘图
set(h2,'fontsize',15,'color','r');
xlabel('距离/m');
ylabel("浓度");
a=fit(1);
%以上是问题一

 X=[-10 ;0;10;0];Y=[0;-10;0;10]; 
C=[158.198;114.942;151.557;158.198];
 figure;
nihe=@(beta,xx)xx(:,4).*beta(3).*exp(xx(1,3).*((xx(:,1)-beta(1)).^2+(xx(:,2)-beta(2)).^2).^0.5);%定义需要拟合的匿名函数
%nihe=@(beta,xx) beta(3).*exp(xx(1,3).*((xx(:,1)-beta(1)).^2+(xx(:,2)-beta(2)).^2).^0.5)...
   % +beta(6).*exp(xx(1,3).*((xx(:,1)-beta(4)).^2+(xx(:,2)-beta(5)).^2).^0.5)...
    %+beta(7)*(beta(3).*exp(xx(1,3).*((xx(:,1)-beta(1)).^2+(xx(:,2)-beta(2)).^2).^0.5))...
    %.*beta(6).*exp(xx(1,3).*((xx(:,1)-beta(4)).^2+(xx(:,2)-beta(5)).^2).^0.5);
w=ones(length(X),1);
xx=[X Y a*ones(length(X),1) w];
beta0=[-1,1,190];
Ci=C;
[beta1,ri,j]=nlinfit(xx,Ci,nihe,beta0);
betaci=nlparci(beta1,ri,'jacobian',j);%计算回归系数的置信区间
[yhat,delta]=nlpredci(nihe,xx,beta1,ri,'jacobian',j);%计算C的预测值及置信区间半径
R1=1-sum((Ci-yhat).^2)/sum((Ci-mean(Ci)).^2);
k=1;


for kappa=0.425:0.1:1.025
 ll=kappa*(sum(w.*(Ci-beta1(3).*exp(a.*((xx(:,1)-beta1(1)).^2+(xx(:,2)-beta1(2)).^2).^0.5)).^2)/((1/4)*sum(w)))^0.5;
  w2=1./abs(ri);
 w2(abs(ri)<ll)=1./ll; %设置门限值
 w2=w2./norm(w2)*2;%归一化
 C=Ci.*w2;
xx=[X Y a*ones(length(X),1) w2];
beta0=[-1,1,190];
[beta,r,j]=nlinfit(xx,C,nihe,beta0);
betaci=nlparci(beta,r,'jacobian',j);%计算回归系数的置信区间
[yhat,delta]=nlpredci(nihe,xx,beta,r,'jacobian',j);%计算C的预测值及置信区间半径
R2(k)=1-sum((C-yhat).^2)/sum((C-mean(C)).^2);%相关指数
r_sum(k)=sum(r.^2);%残差和
lo(:,k)=[beta(1);beta(2)];%位置
cc(:,k)=beta(3);%浓度
ww(:,k)=w2;%权重
k=k+1;
end


%以下是绘图
beta=[lo(:,1);cc(1)];


xxx=-15:0.2:15;
yyy=-15:0.2:15;
[XX,YY]=meshgrid(xxx,yyy);
Z=beta(3).*exp(a.*((XX-beta(1)).^2+(YY-beta(2)).^2).^0.5);
ph=mesh(XX,YY,Z);
set(gca,'fontsize',15);
xlabel('x/m');
ylabel('y/m');
zlabel('C/浓度');
beta=floor(beta*100)/100;
str=['(' num2str(beta(1)) ',' num2str(beta(2)) ',' num2str(beta(3)) ')'];

text(beta(1)+0.2,beta(2)+0.3,beta(3),{str},'fontsize',18);
figure;
[D,h]=contour(XX,YY,Z);
clabel(D,h);
set(gca,'fontsize',12);
xlabel('x/m');
ylabel('y/m');
yhat=yhat./w2;
yhat=floor(yhat*100)/100;
for i=1:4
    str=['\leftarrow' '(' num2str(X(i)) ',' num2str(Y(i)) ',' num2str(yhat(i)) ')'];
hh=text(X(i),Y(i),str);
set(hh,'fontsize',15,'color',[0.6 0.6 0.6]);
box on;
end

str=['\leftarrow' '(' num2str(beta(1)) ',' num2str(beta(2)) ',' num2str(beta(3)) ')'];
hh=text(beta(1),beta(2),str);
set(hh,'fontsize',15,'color',[0.6 0.6 0.6]);
box on;





\end{lstlisting}
以下为问题二方案二的求解:
\begin{lstlisting}
  
  x=[-10 ;0;10;0;-10];y=[0;-10;0;10;0];
 C=[158.198;114.942;151.557;158.198];
 b=-0.02;


ff=@(xyc) [xyc(3)*exp(b*((x(1)-xyc(1))^2+(y(1)-xyc(2))^2)^0.5)-C(1);
%xyc(3)*exp(b*((x(2)-xyc(1))^2+(y(2)-xyc(2))^2)^0.5)-C(2);
xyc(3)*exp(b*((x(3)-xyc(1))^2+(y(3)-xyc(2))^2)^0.5)-C(3);
xyc(3)*exp(b*((x(4)-xyc(1))^2+(y(4)-xyc(2))^2)^0.5)-C(4);]

rr=fsolve(ff,[1 1 180]);
h=@(xyc) [rr(3)*exp(b*((xyc(1)*cos(xyc(2))+1)*(x(2)-rr(1))^2+(xyc(1)*sin(xyc(2))+1)*(y(2)-rr(2))^2)^0.5)-C(2);...
rr(3)*exp(b*((xyc(1)*cos(xyc(2))+1)*(x(1)-rr(1))^2+(xyc(1)*sin(xyc(2))+1)*(y(1)-rr(2))^2)^0.5)-C(1);]
rrr=fsolve(h,[1 pi/2])
te=rrr(2);%得出方向
g=@(xyc) [xyc(3)*exp(b*((xyc(4)*cos(te)+1)*(x(1)-xyc(1))^2+(xyc(4)*sin(te)+1)*(y(1)-xyc(2))^2)^0.5)-C(1);
xyc(3)*exp(b*((xyc(4)*cos(te)+1)*(x(2)-xyc(1))^2+(xyc(4)*sin(te)+1)*(y(2)-xyc(2))^2)^0.5)-C(2);
xyc(3)*exp(b*((xyc(4)*cos(te)+1)*(x(3)-xyc(1))^2+(xyc(4)*sin(te)+1)*(y(3)-xyc(2))^2)^0.5)-C(3);
xyc(3)*exp(b*((xyc(4)*cos(te)+1)*(x(4)-xyc(1))^2+(xyc(4)*sin(te)+1)*(y(4)-xyc(2))^2)^0.5)-C(4)];
re=fsolve(g,[1 1 180 1]);%得出污染源信息
%以下是绘图
beta=re;
xxx=-15:0.2:15;
yyy=-15:0.2:15;
[XX,YY]=meshgrid(xxx,yyy);
ff=@(xi,yi) beta(3).*exp(b.*((beta(4).*cos(te)+1).*(xi-beta(1)).^2+(beta(4).*sin(te)+1).*(yi-beta(2)).^2).^0.5);
Z=ff(XX,YY);
ph=mesh(XX,YY,Z);
set(gca,'fontsize',15);
xlabel('x/m');
ylabel('y/m');
zlabel('C/浓度');
beta=floor(beta*100)/100;
str=['(' num2str(beta(1)) ',' num2str(beta(2)) ',' num2str(beta(3)) ')'];

text(beta(1)+0.2,beta(2)+0.3,beta(3),{str},'fontsize',18);
figure;
[D,h]=contour(XX,YY,Z);
clabel(D,h);
set(gca,'fontsize',12);
xlabel('x/m');
ylabel('y/m');

yhat=ff(x,y);
for i=1:4
    str=['\leftarrow' '(' num2str(x(i)) ',' num2str(y(i)) ',' num2str(yhat(i)) ')'];
hh=text(x(i),y(i),str);
set(hh,'fontsize',15,'color',[0.6 0.6 0.6]);
box on;
end

str=['\leftarrow' '(' num2str(beta(1)) ',' num2str(beta(2)) ',' num2str(beta(3)) ')'];
hh=text(beta(1),beta(2),str);
set(hh,'fontsize',15,'color',[0.6 0.6 0.6]);
box on;

    \end{lstlisting}
以下是对方案二的灵敏度分析:
\begin{lstlisting}
  x=[-10 ;0;10;0;-10];y=[0;-10;0;10;0];
  C=[158.198;114.942;151.557;158.198];
  b=-0.02;
 i= 1.5854;
 k=1;
 for te=i-1:0.01:i+1
 g=@(xyc) [xyc(3)*exp(b*((xyc(4)*cos(te)+1)*(x(1)-xyc(1))^2+(xyc(4)*sin(te)+1)*(y(1)-xyc(2))^2)^0.5)-C(1);
 xyc(3)*exp(b*((xyc(4)*cos(te)+1)*(x(2)-xyc(1))^2+(xyc(4)*sin(te)+1)*(y(2)-xyc(2))^2)^0.5)-C(2);
 xyc(3)*exp(b*((xyc(4)*cos(te)+1)*(x(3)-xyc(1))^2+(xyc(4)*sin(te)+1)*(y(3)-xyc(2))^2)^0.5)-C(3);
 xyc(3)*exp(b*((xyc(4)*cos(te)+1)*(x(4)-xyc(1))^2+(xyc(4)*sin(te)+1)*(y(4)-xyc(2))^2)^0.5)-C(4)];
 re=fsolve(g,[1 1 180 1]);
 paint(k,:)=[te re];
 k=k+1;
 end
 %以下是绘图
 plot(paint(:,1),paint(:,2));
 set(gca,'fontsize',12);
 title('x_0的变化情况');
 xlabel('\theta/弧度');
 ylabel("x_0/m");
 figure;
 plot(paint(:,1),paint(:,3));
 set(gca,'fontsize',12);
 title('y_0的变化情况');
 xlabel('\theta/弧度');
 ylabel("y_0/m");
 figure;
 plot(paint(:,1),paint(:,4));
 set(gca,'fontsize',12);
 title('污染源浓度的变化情况');
 xlabel('\theta/弧度');
 ylabel("污染源浓度");
 figure;
 plot(paint(:,1),paint(:,5));
 set(gca,'fontsize',12);
 title('作用强度的变化情况');
 xlabel('\theta/弧度');
 ylabel("作用强度");
 
  nn=logical(paint(:,1)>=i-0.25)&logical(paint(:,1)<=i+0.25);
  h=paint(find(nn==1),:);
  fin=h(1:3:end,:);
  %导出数据
  xlswrite("out.xlsx",fin);
  
 
\end{lstlisting} 
\end{document}